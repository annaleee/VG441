\documentclass[a4paper,12pt]{journal}
\usepackage[dvipsnames, svgnames, x11names]{xcolor} 
\usepackage{amsmath}
\usepackage{amssymb}
\usepackage[margin=2.5cm]{geometry}
\usepackage{graphics}
\usepackage{ulem}
\usepackage{setspace}
\usepackage{listings}
\usepackage{algorithm}  
\usepackage{algpseudocode}  
\usepackage{amsmath}  
\usepackage{xcolor}
\usepackage[greek,english]{babel}
\usepackage{chemformula}
\usepackage{wrapfig}
\usepackage{multirow}
\usepackage{booktabs}
\usepackage{fancyhdr}
\usepackage{pgfplots}
\usepackage{tikz}
\pagestyle{fancy}
\usetikzlibrary{math}
\rmfamily
\fancyhf{}
\fancyfoot[R]{\thepage}
\fancyhead[R]{VG441 HW3}
\title{VG441 Problem Set 3}
\author{Anna Li \\Student ID: 518370910048}
\date{\today}
\lstset{
	columns=fixed,     
	numbers=left,                                        % 在左侧显示行号
	numberstyle=\tiny\color{gray},                       % 设定行号格式
	frame=none,                                          % 不显示背景边框
	backgroundcolor=\color[RGB]{245,245,244},            % 设定背景颜色
	keywordstyle=\color[RGB]{40,40,255},                 % 设定关键字颜色
	numberstyle=\footnotesize\color{darkgray},           
	commentstyle=\it\color[RGB]{0,96,96},                % 设置代码注释的格式
	stringstyle=\ttfamily\slshape\color[RGB]{128,0,0},   % 设置字符串格式
	showstringspaces=false,                              % 不显示字符串中的空格                                        % 设置语言
}
\begin{document}
	\maketitle
	\section*{Problem 1}
	\subsection*{1. Formulate the set cover problem as a MILP}
	\textbf{Decision Variables:}\\
	Our choices of sets: $x_i\in \{0,1\},\quad i\in\{1,2,...,m\}$.\\
	elements and sets: $s_{mn}\in\{0,1\}$, if set m has element n of V, then $s_{mn}=1$, otherwise $s_{mn}=0$\\
	\textbf{Objective:}\\
	Minimize $\sum_{m}^{1}x_i$\\
	\textbf{Constraints:}\\
	$(SX)_{n}\geq 1$ for $\forall n$\\
	$\sum_{1}^{m}x_i\geq 1$
	\subsection*{2. Solve the problem on Page 4 of LEC015 using Gurobi}
	After running the gurobi codes, we get the solution that:
	\begin{lstlisting}
Thread count: 4 physical cores, 8 logical processors, using up to 8 threads
Optimize a model with 8 rows, 5 columns and 13 nonzeros
Model fingerprint: 0x74c2bfc5
Variable types: 0 continuous, 5 integer (5 binary)
Coefficient statistics:
Matrix range     [1e+00, 1e+00]
Objective range  [1e+00, 1e+00]
Bounds range     [1e+00, 1e+00]
RHS range        [1e+00, 1e+00]
Found heuristic solution: objective 4.0000000
Presolve removed 8 rows and 5 columns
Presolve time: 0.00s
Presolve: All rows and columns removed

Explored 0 nodes (0 simplex iterations) in 0.00 seconds
Thread count was 1 (of 8 available processors)

Solution count 1: 4 

Optimal solution found (tolerance 1.00e-04)
Best objective 4.000000000000e+00, best bound 4.000000000000e+00, gap 0.0000%

Variable            X 
-------------------------
decision var[1]            1 
decision var[2]            1 
decision var[3]            1 
decision var[4]            1 

Process finished with exit code 0
	
	\end{lstlisting}
Therefore, the solution is:
we choose set 1, 3, 4, 5
\section*{Problem 2}
We want to prove that greedy algorithm provides the optimal solution for the Fractional Knaspack Problem.\\
Suppose there are n items, each item i has a value $v_i$ and size $s_i$\\
The capacity of backpack is B\\
We could use contradiction to prove:\\
Assume that there exists a solution SOL of Fractional Knaspack Algorithm which is not optimal, \\
Suppose $\text{SOL}=\{l_1,l_2,...,l_n\}$, and the optimal solution is $\{o_1,o_2,...,o_n\}$, $s_i$ and $o_i$ mean whether the ith item is chosen, and the items are ordered by $\frac{\text{value}}{\text{size}}$\\
According to the definition of optimal solution:
$$\Rightarrow \sum_{i=1}^{n}l_iv_i<\sum_{i=1}^{n}o_iv_i$$
According to the principal of greedy algorithm, there exists a certain a .For all $i\geq a$,  $l_i\geq o_i$ .\\

	\begin{align}
		\text{Value}(SOL)-\text{Value}(Optimal)&=\sum_{i=1}^{n}(l_i-o_i)(\frac{v_i}{s_i})s_i\\
		&=\sum_{i=1}^{a}(1-o_i)(\frac{v_i}{s_i})s_i+\sum_{i=a+1}^{n}(-o_i)(\frac{v_i}{s_i})s_i
	\end{align}
Since for roughly equal size, value of every unit of size of optimal solution is less than the value of SOL,if SOL is not equal to the optimal solution, than the value of SOL is greater than the value of optimal solution, which contradicts. Therefore, optimal solution must be equal to the greedy algorithm solution. \\

\end{document}
