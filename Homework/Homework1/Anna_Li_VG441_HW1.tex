\documentclass[a4paper,12pt]{journal}
\usepackage[dvipsnames, svgnames, x11names]{xcolor} 
\usepackage{amsmath}
\usepackage{amssymb}
\usepackage[margin=2.5cm]{geometry}
\usepackage{graphics}
\usepackage{ulem}
\usepackage{setspace}
\usepackage{listings}
\usepackage{algorithm}  
\usepackage{algpseudocode}  
\usepackage{amsmath}  
\usepackage{xcolor}
\usepackage[greek,english]{babel}
\usepackage{chemformula}
\usepackage{wrapfig}
\usepackage{multirow}
\usepackage{booktabs}
\usepackage{fancyhdr}
\usepackage{pgfplots}
\usepackage{tikz}
\pagestyle{fancy}
\rmfamily
\fancyhf{}
\fancyfoot[R]{\thepage}
\fancyhead[R]{VG441 HW1\&2}
\title{VG441 Problem Set 1}
\author{Anna Li \\Student ID: 518370910048}
\date{\today}
\lstset{
	columns=fixed,     
	numbers=left,                                        % 在左侧显示行号
	numberstyle=\tiny\color{gray},                       % 设定行号格式
	frame=none,                                          % 不显示背景边框
	backgroundcolor=\color[RGB]{245,245,244},            % 设定背景颜色
	keywordstyle=\color[RGB]{40,40,255},                 % 设定关键字颜色
	numberstyle=\footnotesize\color{darkgray},           
	commentstyle=\it\color[RGB]{0,96,96},                % 设置代码注释的格式
	stringstyle=\ttfamily\slshape\color[RGB]{128,0,0},   % 设置字符串格式
	showstringspaces=false,                              % 不显示字符串中的空格                                        % 设置语言
}
\begin{document}
	\maketitle
	\section*{Problem 1}
	$$\theta^TX^TX\theta=\sum_{i}(X^TX)_{i,i}\theta_i^2+\sum_{i\not =j}(X^TX)_{i,j}\theta_i\theta_j$$
	$$\Rightarrow \frac{d(\theta^TX^TX\theta)}{d\theta_i}=2(X^TX)_{i,i}\theta_i+\sum_{i\not j}(X^TX)_{i,j}\theta_j+\sum_{i\not j}(X^TX)_{j,i}\theta_j$$
	Since $X^TX$ is symmetric, we could conclude that:
	$$\Rightarrow \frac{d(\theta^TX^TX\theta)}{d\theta}=2(X^TX)\theta$$
	\section*{Problem 2}
	\subsection*{GBM}
	\subsubsection*{Iteration}
	In the first step, we get PR0, and start to calculate the deviance for each dimension. \\
	Deviance of "Age $>$35" = $1.21\times 10^7$\\
	Deviance of "Home Owner=Yes" = $1.27\times 10^7$\\
	Deviance of "Car Owner=Yes" = $2.85\times 10^7$\\
	Deviance of "Having Kids=Yes" = $1.21\times 10^7$\\
	Therefore, we build the tree like:
	\begin{center}
		\begin{tabular}{c c c}
			&&10000\\
			&Car Owner = Yes&\\
			&&8000\\
			Having Kids = Yes&&\\
			&&5000\\
			&Home Owner&\\
			&&500\\
		\end{tabular}
	\end{center}
	In the second step, we get PR1, and start to calculate the deviance and find:\\
	Deviance of "Age $>$35" = $9.8\times 10^6$\\
	Deviance of "Home Owner=Yes" = $2.3\times 10^7$\\
	Deviance of "Car Owner=Yes" = $1.027\times 10^7$\\
	Deviance of "Having Kids=Yes" = $9.8\times 10^6$\\
	Therefore, the tree is the same as the last stage\\
	\subsubsection*{Results}
	Finally, we run GBM o paper and get the table that:
	\begin{center}
		\begin{tabular}{|c|c|c|c|c|c|}
			\hline
			F0&PR0&F1&PR1&F2&PR2\\\hline
			5875&4125&6287.5&3712.5&6658.75&3341.25\\\hline
			5875&-5375&5337.5&-4837.5&4853.75&-4353.75\\\hline
			5875&2125&5787.5&1912.5&6278.75&1721.25\\\hline
			5875&-875&6087.5&-787.5&5708.75&-708.75\\\hline
		\end{tabular}
	\end{center}
	\subsection*{XGBM}
	\subsubsection*{Iteration}
	In the first step, we get PR0, and start to calculate the ss for each dimension. \\
	SS of "Age $>$35" = $2.6\times 10^7$\\
	SS of "Home Owner=Yes" = $2.16\times 10^7$\\
	SS of "Car Owner=Yes" = $1.27\times 10^7$\\
	SS of "Having Kids=Yes" = $2.6\times 10^7$\\
	Therefore, we build the tree like:
	\begin{center}
		\begin{tabular}{c c c}
			&&10000\\
			&Car Owner = Yes&\\
			&&8000\\
			Having Kids = Yes&&\\
			&&5000\\
			&Home Owner&\\
			&&500\\
		\end{tabular}
	\end{center}
	In the second step, we get PR1, and start to calculate the SS and find:\\
	SS of "Age $>$35" = $2.35\times 10^7$\\
	SS of "Home Owner=Yes" = $1.96\times 10^7$\\
	SS of "Car Owner=Yes" = $1.15\times 10^7$\\
	SS of "Having Kids=Yes" = $2.35\times 10^7$\\
	Therefore, the tree is the same as the last stage\\
	\subsubsection*{Results}
	Finally, we run XGBM o paper and get the table that:
	\begin{center}
		\begin{tabular}{|c|c|c|c|c|c|}
			\hline
			F0&PR0&F1&PR1&F2&PR2\\\hline
			5875&4125&6081.25&3918.75&6277.1875&3722.8125\\\hline
			5875&-5375&5606.25&-5106.25&5350.9375&-4850.9375\\\hline
			5875&2125&5981.25&2018.75&6082.1875& 1917.8125\\\hline
			5875&-875&5831.25&-831.25&5789.6875&-789.6875\\\hline
		\end{tabular}
	\end{center}

\end{document}
