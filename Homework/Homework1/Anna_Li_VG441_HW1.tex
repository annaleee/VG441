\documentclass[a4paper,12pt]{journal}
\usepackage[dvipsnames, svgnames, x11names]{xcolor} 
\usepackage{amsmath}
\usepackage{amssymb}
\usepackage[margin=2.5cm]{geometry}
\usepackage{graphics}
\usepackage{ulem}
\usepackage{setspace}
\usepackage{listings}
\usepackage{algorithm}  
\usepackage{algpseudocode}  
\usepackage{amsmath}  
\usepackage{xcolor}
\usepackage[greek,english]{babel}
\usepackage{chemformula}
\usepackage{wrapfig}
\usepackage{multirow}
\usepackage{booktabs}
\usepackage{fancyhdr}
\usepackage{pgfplots}
\usepackage{tikz}
\pagestyle{fancy}
\rmfamily
\fancyhf{}
\fancyfoot[R]{\thepage}
\fancyhead[R]{VG441 HW1\&2}
\title{VG441 Problem Set 1}
\author{Anna Li \\Student ID: 518370910048}
\date{\today}
\lstset{
	columns=fixed,     
	numbers=left,                                        % 在左侧显示行号
	numberstyle=\tiny\color{gray},                       % 设定行号格式
	frame=none,                                          % 不显示背景边框
	backgroundcolor=\color[RGB]{245,245,244},            % 设定背景颜色
	keywordstyle=\color[RGB]{40,40,255},                 % 设定关键字颜色
	numberstyle=\footnotesize\color{darkgray},           
	commentstyle=\it\color[RGB]{0,96,96},                % 设置代码注释的格式
	stringstyle=\ttfamily\slshape\color[RGB]{128,0,0},   % 设置字符串格式
	showstringspaces=false,                              % 不显示字符串中的空格                                        % 设置语言
}
\begin{document}
	\maketitle
	\section*{Problem 1}
		$$\theta^TX^TX\theta=\sum_{i}(X^TX)_{i,i}\theta_i^2+\sum_{i\not =j}(X^TX)_{i,j}\theta_i\theta_j$$
	$$\Rightarrow \frac{d(\theta^TX^TX\theta)}{d\theta_i}=2(X^TX)_{i,i}\theta_i+\sum_{i\not j}(X^TX)_{i,j}\theta_j+\sum_{i\not j}(X^TX)_{j,i}\theta_j$$
	Since $X^TX$ is symmetric, we could conclude that:
	$$\Rightarrow \frac{d(\theta^TX^TX\theta)}{d\theta}=2(X^TX)\theta$$
	\section*{Problem 2}
	\subsection*{Iterations}
	Firstly, we get PR0 = \[4125,-5375,2125,-875\] and get the decision tree like:
	\begin{center}
		\begin{tabular}{c c c}
			&&4125\\
			&Car Owner&\\
			&&2125\\
			Age$<$35&&\\
			&&-875\\
			&Home Owner&\\
			&&-5375\\
		\end{tabular}
	\end{center}
Secondly, we get PR1=\[3712.5,-4837.5,1912.5,-787.5\] and get the decision tree like:
	\begin{center}
	\begin{tabular}{c c c}
		&&3712.5\\
		&Car Owner&\\
		&&1912.5\\
		Age$<$35&&\\
		&&-787.5\\
		&Home Owner&\\
		&&-4837.5\\
	\end{tabular}
\end{center}
	\subsection*{Results}
	Finally, we run GBM o paper and get the table that:
	\begin{center}
		\begin{tabular}{|c|c|c|c|c|c|}
			\hline
			F0&PR0&F1&PR1&F2&PR2\\\hline
			5875&4125&6287.5&3712.5&6658.75&3341.25\\\hline
			5875&-5375&5337.5&-4837.5&4853.75&-4353.75\\\hline
			5875&2125&5787.5&1912.5&6278.75&1721.25\\\hline
			5875&-875&6087.5&-787.5&5708.75&-708.75\\\hline
		\end{tabular}
	\end{center}
	\subsection*{XGBM}
		\subsection*{Iterations}
	Firstly, we get PR0 = \[4125,-5375,2125,-875\] and get the decision tree like:
	\begin{center}
		\begin{tabular}{c c c}
			&&2125,4125\\
			Age$<$35&&\\
			&&-875\\
			&Home Owner&\\
			&&-5375\\
		\end{tabular}
	\end{center}
	Secondly, we get PR1=\[3916.7,-5106.25,1916.7,-831.25\] and get the decision tree like:
	\begin{center}
		\begin{tabular}{c c c}
			&&3916.7,1916.7\\
			Age$<$35&&\\
			&&-831.25\\
			&Home Owner&\\
			&&-5106.25\\
		\end{tabular}
	\end{center}
	\subsection*{Results}
	Finally, we run XGBM o paper and get the table that:
	\begin{center}
		\begin{tabular}{|c|c|c|c|c|c|}
			\hline
			F0&PR0&F1&PR1&F2&PR2\\\hline
			5875&4125&6083.3&3916.7&6277.7&3277.25\\\hline
			5875&-5375&5606.25&-5106.25&5350.9375&-4850.9375\\\hline
			5875&2125&6083.3&1916.7&6277.7& 1722.25\\\hline
			5875&-875&5831.25&-831.25&5789.6875&-789.6875\\\hline
		\end{tabular}
	\end{center}

\end{document}
